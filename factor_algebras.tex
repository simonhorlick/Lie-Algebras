\section{Factor algebras, homeomorphisms and representations}

\subsection{Some linear algebra}
Let $V$ be a vector space over a field $k$ and $W$ a subspace. A \emph{coset} $v + W$ is the subset of $V$ of the form
\[
  v + W = \{ v + w | w \in W \}.
\]
One knows that $v + W = v' + W \iff v - v' \in W$. Let $V/W$ be the set of all cosets in $V$. Give $V/W$ two operations (vector addition and scalar multiplication) by
\begin{align*}
  (u + W) + (v + W) &:= u + v + W\\
     \lambda(v + W) &:= \lambda v + W
\end{align*}
for all $\lambda \in k$ and all $u, v \in V$.

These two operations are well defined. For example, if $v + W = v' + W$ (so that $v - v' \in W$) then
\begin{align*}
  \lambda(v' + W) &= \lambda(v - (v - v')) + W\\
                  &= \lambda v - \lambda(v - v') + W\\
                  &= \lambda v + W.
\end{align*}
The zero vector of $V/W$ is the zero coset $0 + W = W$. One checks that all axioms of a vector space are satisfied and so $V/W$ is a vector space over $k$.

Suppose $\text{dim } V = n$ is finite, then $\text{dim } W = m \leq n$. Let $w_1, \dots, w_m$ be a basis for $W$. By basis extension theorem we can extend this up to a basis of $V$. Let $S = \{ v_1, \dots, v_{n-m}, w_1, \dots, w_m \}$. Any $v \in V$ is a linear combination of vectors in $S$, i.e.
\[
  v = \alpha_1 v_1 + \dots + \alpha_{n-m} v_{n-m} + \beta_1 w_1 + \dots + \beta_m w_m
\]
for $\alpha_i, \beta_i \in k$. Then
\[
  v + W = \alpha_1 (v_1 + W) + \dots + \alpha_{n-m} (v_{n-m} + W)
\]
hence the vectors $v_1 + W, \dots, v_{n-m} + W \in V/W$ span $V/W$. From the definition one deduces that these cosets are linearly independent in $V/W$. So they form a bayesis of $V/W$ and $\text{dim } V/W = n - m = \text{dim } V - \text{dim } W$. We call $V/W$ the \emph{factor space} of $V$ by $W$.

\begin{remark}
  \begin{itemize}
    \item If $W = V$ then $V/W$ is the zero space.
    \item If $W = \{ 0 \}$ then $V/W$ identifies with $V$.
  \end{itemize}
\end{remark}

Let $L$ be a lie algebra over $k$ and $I$ an ideal of $L$. Then the factor space $L/I$ has a bilinear operation $[-,-]$ induced by the lie bracket on $L$. Namely, for $x,y \in L$ we define
\[
  [x + I, y + I] := [x, y] + I.
\]
This definition is correct, for if $x + I = x' + I$ and $y + I = y' + I$ then $x' - x, y' - y \in I$ and hence
\begin{align*}
  [x' + I, y' + I] &= [x', y'] + I\\
                   &= [x + (x' - x) + I, y + (y' - y) + I\\
                   &= [x, y] + [x, y' - y] + [x' - x, y] + [x' - x, y' - y] + I\\
                   &= [x, y] + I.
\end{align*}
We have $[x + I, x + I] = [x, x] + I = I$ ($ = 0 \text{ in } L/I$) so $L/I$ is anticommutative.

Recall $j(x, y, z)$ then
\[
  j(x + I, y + I, z + I) = j(x, y, z) + I = I = 0 \text{ in } L/I.
\]
We conclude that ($L/I, [-,-]$) is a lie algebra. Call it the \emph{factor algebra} of $L$ by $I$.

\begin{definition}
  Let $L$ and $M$ be two lie algebras over $k$. A linear map $\phi : L \to M$ is called a (lie algebra) \emph{homomorphism} if
  \[
    \phi([x, y]) = [\phi(x), \phi(y)]
  \]
  for all $x, y \in L$.
\end{definition}

\begin{example}
  The zero map $x \mapsto 0$ is a lie algebra homomorphism. Since $\phi$ is a linear map, $\text{ker } \phi := \{ x \in L | \phi(x) = 0 \}$ is a subspace of $L$ and $\text{Im } \phi := \{ y \in M | y = \phi(x) for x \in L \}$.
\end{example}

If $\phi$ is bijective then we say that $\phi$ is an \emph{isomorphism} (of lie algebras) and $\phi$ exists, we write $L \cong M$. One studies lie algebras up to isomorphism.

\begin{theorem}[On isomorphisms]
  Let $\phi : L \to M$ be a homomorphism of lie algebras. Then
  \begin{enumerate}
    \item $\text{ker }\phi$ is an ideal of $L$;
    \item $\text{Im }\phi$ is a subalgebra of $M$;
    \item $L/\text{ker }\phi \cong \text{Im }\phi$ as lie algebras.
  \end{enumerate}
\end{theorem}

\begin{proof}

\end{proof}

If $V$ is a vector space and $L$ is a lie algebra, both over $k$. Then a lie algebra homomorphism $\rho : L \to \mathfrak{gl}(V)$ is called a \emph{representation} of $L$ (in $\mathfrak{gl}(V)$). If $\text{dim } V$ is finite then $\rho$ is called \emph{finite dimensional}. Then $\text{dim } V$ is called the \emph{degree} of $\rho$, denoted $\text{deg }\rho$. We say that $\rho : L \to \mathfrak{gl}(V)$ is \emph{faithful} if $\text{ker }_\rho = \{ 0 \}$. Then $\rho(L) \cong L/\text{ker }\rho \cong L$. So $L$ identifies with a subalgebra of $\mathfrak{gl}(V)$.

