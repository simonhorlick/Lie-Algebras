\section{Introduction}

A Lie group is a group + a manifold (or a variety)
\begin{itemize}
  \item $e \in G$ identity element
  \item product and $x \mapsto x^{-1}$ are continuous operations
  \item $T_e(G)$, the tangent space at $e$, has a certain bilinear operation which is not associative.
\end{itemize}

\subsection{Algebras}

Let $k$ be a field (e.g. $k=\mathbb{C}$, or $k=\overline{\mathbb{F}_p}$).

\begin{definition}
  A vector space $A$ over $k$ is a \emph{$k$-algebra} if $A$ carries a binary operation $A \times A \to A$ which is $k$-bilinear:
  \begin{align*}
    (\lambda_1 x_1 + \lambda_2 x_2) \cdot y &= \lambda_1 x_1 \cdot y + \lambda_2 x_2 \cdot y\\
    x \cdot (\lambda_1 y_1 + \lambda_2 y_2) &= \lambda_1 x \cdot y_1 + \lambda_2 x \cdot  y_2
  \end{align*}
  $\forall x_i, y_i, x, y \in A, \forall \lambda_i \in k$.
\end{definition}

\begin{example}
  If $V$ is a vector space over $k$ then we give $V$ an algebra structure by:
  \[
    v \cdot w = 0 \qquad \forall v,w \in V.
  \]
  Then $V$ becomes an algebra with zero multiplication.
\end{example}

\begin{example}
  $A=M_n(k)$ is a $k$-algebra with respect to the matrix product. Bilinearity of the matrix product is proved in linear algebra.
\end{example}

\begin{remark}
  Let $A$ be a $k$-vector space with basis $\{v_1, \dots, v_n\}$ (so $\text{dim } A = n$). To each pair $(v_i,v_j)$ we assign a vector $w_{ij} \in A$. Then we define a binary operation on $A$ by the following rule:
  \[
    \left( \sum_{i=1}^n \lambda_i v_i \right) \cdot \left( \sum_{j=1}^n \mu_j v_j \right) := \sum_{i=1}^n \sum_{j=1}^n \lambda_i \mu_j w_{ij}.
  \]
  It is straight forward to check that this product is bilinear.
\end{remark}

\begin{definition}
  An algebra $A$ is called \emph{alternative} if
  \begin{align*}
    (x \cdot x) \cdot y &= x \cdot (x \cdot y)\\
    y \cdot (x \cdot x) &= (y \cdot x) \cdot x \qquad \forall x,y \in A.
  \end{align*}
\end{definition}
Examples include $\mathbb{O}$, the algebra of Cayley octonions ($\text{dim } \mathbb{O} = 8$).

\begin{definition}
  An algebra $A$ is called \emph{anticommutative} if
  \begin{equation}\label{eqn:anticommutative}
    x \cdot x = 0 \quad \forall x \in A.
  \end{equation}
  Setting $x+y$ for $x$ gives
  \[
    0 = (x+y) \cdot (x+y) = x \cdot x + x \cdot y + y \cdot x + y \cdot y
  \]
  by bilinearity, where $x \cdot x = 0$ and $y \cdot y = 0$. Hence
  \begin{equation}\label{eqn:anticommutative2}
    x \cdot y = -y \cdot x
  \end{equation}
  in any anticommutative algebra. Put $y=x$, then we get $x \cdot x = -x \cdot x$ or $2 x \cdot x = 0$. So if $\text{char}(k) \not= 2$ (i.e. $2 \not= 0$ in $k$) then Equation (\ref{eqn:anticommutative2}) implies Equation (\ref{eqn:anticommutative}).
\end{definition}

\begin{definition}
  We say that $A$ satisfies the Jacobi identity if
  \[
    x \cdot (y \cdot z) + y \cdot (z \cdot x) + z \cdot (x \cdot y) = 0
  \]
  $\forall x,y,z \in A$. We say that $A$ is a \emph{Lie algebra} if it is anticommutative and satisfies the Jacobi identity.
\end{definition}

\begin{remark}
  Lie algebras are often denoted by lower case gothic characters. The product in a Lie algebra is denoted $[x,y]$ rather than $x \cdot y$. This notation is called the \emph{Lie product} or \emph{Lie bracket}.

  $L$ is a Lie algebra if
  \begin{itemize}
    \item $[x,x] = 0$ and
    \item $[x,[y,z]] + [y,[z,x]] + [z,[x,y]] = 0 \quad \forall x,y,z \in L$.
  \end{itemize}
\end{remark}

\subsection{How to obtain Lie algebras from associative algebras}

Let $A$ be an associative algebra over $k$. We give $A$ a new product denoted by $[\cdot,\cdot]$ by,
\[
  [x,y] = x \cdot y - y \cdot x \quad \text{(the commutator product)}.
\]
The algebra obtained is denoted $A^{(-)}$.

\begin{remark}
  If $A$ is commutative then $A^{(-)}$ has zero multiplication.
\end{remark}

\begin{proposition}
  $A^{(-)}$ is a Lie algebra.
\end{proposition}

\begin{proof}
  We have $[x,x] \overset{def}{=} x \cdot x - x \cdot x = 0$ $\forall x \in A^{(-)}$, hence $A^{(-)}$ is anticommutative.

  Now let $x,y,z \in A$.
  \begin{align*}
    [x,[y,z]] + [y,[z,x]] + [z,[x,y]] &\overset{def}{=} x \cdot [y,z] - [y,z] \cdot x\\
    &+ y \cdot [z,x] - [z,x] \cdot y\\
    &+ z \cdot [x,y] - [x,y] \cdot z\\
    &= x \cdot (y \cdot z - z \cdot y) - (y \cdot z - z \cdot y) \cdot x\\
    &+ y \cdot (z \cdot x - x \cdot z) - (z \cdot x - x \cdot z) \cdot y\\
    &+ z \cdot (x \cdot y - y \cdot x) - (x \cdot y - y \cdot x) \cdot z\\
    &= xyz - xzy - yzx + zyx\\
    &+ yzx - yxz - zxy + xzy\\
    &+ zxy - zyx - xyz + yxz\\
    &= 0.
  \end{align*}
  Hence $A^{(-)}$ satisfies the Jacobi identity and thus is a Lie algebra.
\end{proof}

If $A = M_n(k)$ then $A^{(-)}$ has a special name $\textgoth{gl}(n, k)$ the general linear Lie algebra of order $n$ over $k$.

\begin{remark}
  If $G=GL_n(\R)$, a very important Lie group, then $T_e(G) = \textgoth{gl}(n, \R)$.
\end{remark}

\begin{definition}
	Let $V$ be an $n$ dimension vector space over $k$. We define $A(V)$ to be the space of all linear operators $A : V \to V$ on $V$. This is an associative algebra with respect to composition:
	\[
		A \circ B(v) = A(B(v)) \quad \text{for all } v \in V, \text{ for all } A, B \in A(V).
	\]
\end{definition}

\begin{definition}
	The Lie algebra $A(V)^{(-)}$ is denoted $\textgoth{gl}(V)$, the \emph{general linear Lie algebra of $V$}. We have
	\[
		[A, B] = A \circ B - B \circ A \quad \text{for all } A, B \in \textgoth{gl}(V).
	\]
\end{definition}

\subsection{Subalgebras and Ideals}
\begin{definition}
	Let $A$ be an algebra over $k$. A subspace $B$ of $A$ is a \emph{subalgebra} if $x \cdot y \in B$ for all $x, y \in B$.
\end{definition}

\begin{example}
	\begin{itemize}
		\item (Lazy example) If $A$ has zero multiplication, then any subspace $V$ of $A$ is a subalgebra. Indeed, if $x, y \in V$, then $x \cdot y = 0 \in V$ (since $V$ is a subspace).
		\item If $A$ is anticommutative, then for all $x \in A$ the $k$-span $kx$ is a subalgebra of $A$:
		\[
			\lambda x \circ \mu x = \lambda\mu(\underbrace{x \circ x}_{= 0}) = 0 \in kx.
		\]
	\end{itemize}
\end{example}

\begin{definition}
	A subspace $I$ of an algebra $A$ is a \emph{left ideal} if $a \cdot x \in I$ for all $a \in A$ and for all $x \in I$. \emph{Right ideals} are defined similarly.
\end{definition}

\begin{example}
	If $A = M_n(k)$ then
	\[
		I := \left\{
		\begin{pmatrix}
			\alpha_1 & 0 & \ldots & 0 \\
			\alpha_2 & 0 & \ldots & 0 \\
			\vdots	 & \vdots & \ddots & \vdots \\
			\alpha_n & 0 & \ldots & 0
		\end{pmatrix}
		\ |\ \alpha_i \in k, i \in \{1, \ldots n\}\right\}
	\]
	is a left ideal of $A$.
\end{example}
